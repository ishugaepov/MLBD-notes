В практических задачах ML часто приходится работать с “большими данными”, т.е. данными которые невозможно или слишком долго обрабатывать на одной машине. \newline

Такие данные возникают во многих областях: интернет, финансы, датчики, NLP… \newline

На стыке ML и CS возникла субдисциплина “Машинное обучение на больших данных” (Large Scale Machine Learning). \newline

Исследования по LSML ведутся в следующих основных направлениях:
\begin{itemize}
    \item как существующие алгоритмы ML ускорить или перенести на распределенные системы;
    \item разработка новых алгоритмов ML, изначально предназначенных для больших данных;
    \item разработка архитектур и моделей вычислений, подходящих для LSML;
\end{itemize}
$ $\\ 
Цели этого курса:
\begin{itemize}
    \item объяснить, зачем нужно LSML, что это за область, в чем отличие от традиционного ML;
    \item дать теоретические знания об основных алгоритмах ML, работающих с большими данными;
    \item научить применять существующие программы для LSML на практике.
    \item научить писать алгоритмы ML под архитектуры Hadoop, Spark, GraphLab.
\end{itemize}
$ $\\ 
Курс состоит из трех частей:
\begin{enumerate}
    \item Методы и системы обработки больших данных;
    \item Машинное обучение на больших данных;
    \item Проведение онлайн экспериментов.
\end{enumerate}