\documentclass[openany,12pt]{book}
\usepackage{ucs}
\usepackage[utf8x]{inputenc}
\usepackage[russian]{babel}
\usepackage{mathtools}
\usepackage{tikz}
\usepackage{rotating}
\usetikzlibrary{snakes,automata,backgrounds,petri,fit}
\usetikzlibrary{shapes.geometric}
\usetikzlibrary{positioning}
\usetikzlibrary{arrows}
\usepackage{tikz-3dplot}
\usepackage{listings}
\usepackage{color} 
\usepackage{graphicx}
\usepackage{amssymb}
\usepackage{amsmath}
\usepackage[makeroom]{cancel}
\usepackage{amsthm}
\usepackage{enumerate}
\usepackage[section]{algorithm}
\usepackage{algorithmicx}
\usepackage[noend]{algpseudocode}
\usepackage{textcomp}
\usepackage{import}
\usepackage{hyperref}
\usepackage{listings}
\usepackage[font=footnotesize]{caption}
\usepackage[left=2cm,right=2cm,top=2cm,bottom=2cm,bindingoffset=0cm]{geometry}
\renewcommand{\thealgorithm}{\arabic{section}.\arabic{algorithm}} 
\newtheorem{theorem}{Теорема}[section]
\newtheorem{remark}{Замечание}[section]
\newtheorem{corollary}{Следствие}[section]
\newtheorem{lemma}[theorem]{Лемма}
\newtheorem{definition}{Определение}[section]
\newtheorem{assumption}{Предположение}
\newtheorem{observation}{Наблюдение}
\newtheorem{task}{Задача}[]
\DeclareMathOperator{\rank}{\textbf{rank}}
\DeclareMathOperator{\tr}{tr}
\renewcommand{\O}{\mathcal{O}}

\usepackage[default,scale=0.95]{opensans}
\renewcommand\seriesdefault{l}
\renewcommand\mddefault{m}
\renewcommand\bfdefault{m}% or \renewcommand\bfdefault{m}

\numberwithin{equation}{section}

\usepackage{titlesec}
\newcommand{\sectionbreak}{\clearpage}

\usepackage[parfill]{parskip}

\title{Machine Learning on Big Data}
\author{Шугаепов Ильнур 
\\
\small{VK.com} 
\\
\small{itmo.stud@gmail.com}}
\date{}

\begin{document}
% {\fontfamily{cmss}\selectfont
\maketitle

\tableofcontents

\section*{Введение}
\import{}{introduction.tex}
 
\part{Методы и системы обработки больших данных}
\chapter{Introduction to Hadoop DFS and MapReduce}
\subimport{chapters/hadoop_map_reduce/}{hadoop_map_reduce.tex}


\chapter{Apache Spark}
\subimport{chapters/apache_spark/}{apache_spark.tex}

\chapter{Spark SQL}
\subimport{chapters/spark_sql/}{spark_sql.tex}

\part{Машинное обучение на больших данных}
\chapter{ML in Production (Logging, Model, Features, Deployment)}

\chapter{Logistic Regression (Hashing, Parametric Server)}

\chapter{kNN (LSH)}

\chapter{GBDT}

\chapter{Collaborative Filtering (ALS)}

\chapter{LDA}

\chapter{Spark MLlib Overview}

\chapter{Online Learning}

\chapter{Hyperparameters Tuning + AutoML}

\part{Проведение онлайн экспериментов}
\chapter{How to conduct AB Tests (Experiment Design, Execution, Analysis)}

\chapter{Results Analysis ((Multiple) Hypothesis testing, Sensitivity, Power)}

\chapter{Heterogeneous Treatment Effect}

\nocite{*}

\bibliographystyle{ugost2008ls}
\bibliography{references}

% \subsection{Домашние Задания}
% Может делать практики/демонстрации в виде jupyter тетрадок, которые будут работать с вкшным кластером? С одной стороны хорошо, а с другой было бы еще лучше если бы был кластер, к которому у студентов был бы доступ
% Можно провести какой нибудь анонимный опрос в начале семестра, чтобы прощупать аудиторию
% Лучше все таки сделать ДЗ!
% Писать планы лекций
% Может еще рассказать кратко про luigi и airflow?

% \subsubsection{ML Service}
% Реализовать рекомендательный сервис (с простым web интерфейсом). \\
% Пример: Сервис рекомендации новостей. \\
% Особенности:
% \begin{itemize}
%     \item Для обучения модели использовали открытый датасет Bing News. Новости из этого же датасета используются при демострации работы сервиса.
%     \item Сервис позволяет проводить AB Тесты (есть система логирования и разделения трафика на экспериментальные группы на основе \texttt{user\_id})
% \end{itemize}


% \subsubsection{AB Test}
% Можно будет предоставить часть логов эксперимента, который проводился на реальных пользователях VK. Данные анонимизированы. По логам нужно будет посчитать метрики и оценить стат. значимость результатов.

% }
\end{document}
